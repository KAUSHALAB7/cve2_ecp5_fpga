% ============================================================================
% ECP5-5G FPGA RECIPE BOOK: From Zero to Blinky and CVE2 RISC-V
% Author: Kaushal
% Date: November 5, 2025
% ============================================================================
\documentclass[12pt,a4paper]{article}

% Packages
\usepackage[utf8]{inputenc}
\usepackage[T1]{fontenc}
\usepackage{geometry}
\geometry{a4paper, margin=1in}
\usepackage{times}
\usepackage{microtype}
\usepackage{setspace}
\onehalfspacing
\usepackage{amsmath,amssymb}
\usepackage{graphicx}
\usepackage{booktabs}
\usepackage{array}
\usepackage{float}
\usepackage{hyperref}
\hypersetup{colorlinks=true, linkcolor=blue, urlcolor=blue, citecolor=blue}
\usepackage{xcolor}
\usepackage{listings}

% Listings styles
\definecolor{codebg}{RGB}{247,248,250}
\definecolor{darkgreen}{RGB}{0,120,0}
\lstdefinestyle{bashstyle}{
  backgroundcolor=\color{codebg},
  basicstyle=\ttfamily\small,
  keywordstyle=\color{blue},
  commentstyle=\color{darkgreen},
  stringstyle=\color{magenta},
  showstringspaces=false,
  columns=fullflexible,
  keepspaces=true,
  frame=single,
  framerule=0.5pt,
  breaklines=true,
  prebreak=\raisebox{0ex}[0ex][0ex]{\ensuremath{\hookleftarrow}}
}
\lstdefinestyle{verilogstyle}{
  backgroundcolor=\color{codebg},
  basicstyle=\ttfamily\small,
  keywordstyle=\color{blue},
  commentstyle=\color{darkgreen},
  morekeywords={module,endmodule,wire,reg,always,posedge,negedge,assign,parameter,input,output,logic},
  showstringspaces=false,
  columns=fullflexible,
  keepspaces=true,
  frame=single,
  framerule=0.5pt,
  breaklines=true
}

\title{ECP5-5G FPGA Recipe Book\\\large From Zero to Blinky and CVE2 RISC-V (100\% Open-Source Flow)}
\author{Kaushal}
\date{November 5, 2025}

\begin{document}
\maketitle
\tableofcontents
\newpage

\section{What This Is}
This is a practical, step-by-step “recipe book” to program the Lattice ECP5-5G Versa board from scratch. You’ll learn two end-to-end flows:
\begin{itemize}
  \item A minimal LED blinker (sanity test)\;\;\textbf{10 minutes}
  \item A CVE2 RISC-V SoC running firmware on LEDs\;\;\textbf{30--60 minutes}
\end{itemize}
Everything uses open-source tools: Project Trellis, Yosys, nextpnr-ecp5, sv2v, and OpenOCD.

\section{One-Page Quickstart (Cheat Sheet)}
\subsection{Blinker in 7 Commands}
\begin{lstlisting}[style=bashstyle]
# 1) Create files
cat > top.v <<'EOF'
module top(input wire clk, output wire [7:0] led, output wire [13:0] disp);
  reg [27:0] cnt = 0; always @(posedge clk) cnt <= cnt + 1;
  assign led  = ~cnt[27:20]; // active-LOW LEDs
  assign disp = 14'h3FFF;    // all OFF (active-LOW)
endmodule
EOF
cat > versa.lpf <<'EOF'
LOCATE COMP "clk" SITE "P3"; IOBUF PORT "clk" IO_TYPE=LVDS;
# LEDs (active-LOW)
LOCATE COMP "led[0]" SITE "E16"; IOBUF PORT "led[0]" IO_TYPE=LVCMOS25;
LOCATE COMP "led[1]" SITE "D17"; IOBUF PORT "led[1]" IO_TYPE=LVCMOS25;
LOCATE COMP "led[2]" SITE "D18"; IOBUF PORT "led[2]" IO_TYPE=LVCMOS25;
LOCATE COMP "led[3]" SITE "E18"; IOBUF PORT "led[3]" IO_TYPE=LVCMOS25;
LOCATE COMP "led[4]" SITE "F17"; IOBUF PORT "led[4]" IO_TYPE=LVCMOS25;
LOCATE COMP "led[5]" SITE "F18"; IOBUF PORT "led[5]" IO_TYPE=LVCMOS25;
LOCATE COMP "led[6]" SITE "E17"; IOBUF PORT "led[6]" IO_TYPE=LVCMOS25;
LOCATE COMP "led[7]" SITE "F16"; IOBUF PORT "led[7]" IO_TYPE=LVCMOS25;
# 14-seg display (optional)
LOCATE COMP "disp[0]" SITE "M20"; IOBUF PORT "disp[0]" IO_TYPE=LVCMOS25;
LOCATE COMP "disp[1]" SITE "L18"; IOBUF PORT "disp[1]" IO_TYPE=LVCMOS25;
LOCATE COMP "disp[2]" SITE "M19"; IOBUF PORT "disp[2]" IO_TYPE=LVCMOS25;
LOCATE COMP "disp[3]" SITE "L16"; IOBUF PORT "disp[3]" IO_TYPE=LVCMOS25;
LOCATE COMP "disp[4]" SITE "L17"; IOBUF PORT "disp[4]" IO_TYPE=LVCMOS25;
LOCATE COMP "disp[5]" SITE "M18"; IOBUF PORT "disp[5]" IO_TYPE=LVCMOS25;
LOCATE COMP "disp[6]" SITE "N16"; IOBUF PORT "disp[6]" IO_TYPE=LVCMOS25;
LOCATE COMP "disp[7]" SITE "M17"; IOBUF PORT "disp[7]" IO_TYPE=LVCMOS25;
LOCATE COMP "disp[8]" SITE "N18"; IOBUF PORT "disp[8]" IO_TYPE=LVCMOS25;
LOCATE COMP "disp[9]" SITE "P17"; IOBUF PORT "disp[9]" IO_TYPE=LVCMOS25;
LOCATE COMP "disp[10]" SITE "N17"; IOBUF PORT "disp[10]" IO_TYPE=LVCMOS25;
LOCATE COMP "disp[11]" SITE "P16"; IOBUF PORT "disp[11]" IO_TYPE=LVCMOS25;
LOCATE COMP "disp[12]" SITE "R16"; IOBUF PORT "disp[12]" IO_TYPE=LVCMOS25;
LOCATE COMP "disp[13]" SITE "R17"; IOBUF PORT "disp[13]" IO_TYPE=LVCMOS25;
EOF

# 2) Synthesize, place/route, and pack
yosys -p "synth_ecp5 -top top -json top.json" top.v
nextpnr-ecp5 --um5g-45k --package CABGA381 --json top.json --lpf versa.lpf --textcfg top_out.config
ecppack --svf-rowsize 100000 --svf top.svf top_out.config top.bit

# 3) Program over JTAG (OpenOCD)
openocd -f /usr/share/trellis/misc/openocd/ecp5-versa5g.cfg \
  -c "transport select jtag; init; svf top.svf; exit"
\end{lstlisting}

\subsection{CVE2 SoC in 8 Commands (using existing project)}
\begin{lstlisting}[style=bashstyle]
cd /home/kaushal/cve2_fpga_project
# Build bitstream and program
make -j1 all prog
# (Targets internally run sv2v -> yosys -> nextpnr -> ecppack -> openocd)
\end{lstlisting}

\section{Install the Open-Source Toolchain}
\subsection{Prerequisites (Ubuntu/Debian)}
\begin{lstlisting}[style=bashstyle]
sudo apt-get update
sudo apt-get install -y git build-essential cmake python3 python3-pip \
  libftdi1-2 libftdi1-dev libusb-1.0-0-dev pkg-config \
  texlive-latex-extra # optional (for compiling this guide)
\end{lstlisting}

\subsection{Project Trellis (ECP5 bitstream tools)}
Use your distro packages if recent, or build from source (recommended for latest):
\begin{lstlisting}[style=bashstyle]
# Install Project Trellis (database + tools)
sudo apt-get install -y prjtrellis-tools
# Or from source (snippet)
# git clone https://github.com/YosysHQ/prjtrellis && cd prjtrellis/libtrellis && cmake -DCMAKE_INSTALL_PREFIX=/usr . && make -j$(nproc) && sudo make install
# Database path typically at /usr/share/trellis
\end{lstlisting}

\subsection{Yosys, nextpnr-ecp5, sv2v}
\begin{lstlisting}[style=bashstyle]
# Yosys and nextpnr-ecp5 (use distro or build from source)
sudo apt-get install -y yosys nextpnr nextpnr-ecp5

# sv2v (SystemVerilog -> Verilog)
# Option A: Install from source
# git clone https://github.com/zachjs/sv2v && cd sv2v && cabal update && cabal install
# Option B: Use prebuilt binary if available in PATH
sv2v --version || echo "Install sv2v separately if needed"
\end{lstlisting}

\subsection{OpenOCD (with ECP5 configs)}
\begin{lstlisting}[style=bashstyle]
sudo apt-get install -y openocd
# Config for Versa 5G board typically at:
#   /usr/share/trellis/misc/openocd/ecp5-versa5g.cfg
\end{lstlisting}

\section{Project Structure Templates}
\subsection{Blinker Project}
\begin{lstlisting}[style=bashstyle]
blinker/
  top.v
  versa.lpf
  Makefile
\end{lstlisting}

Minimal Makefile:
\begin{lstlisting}[style=bashstyle]
PROJ = top
DEVICE = --um5g-45k
PACKAGE = --package CABGA381
LPF = versa.lpf

all: $(PROJ).bit

%.json: %.v
	yosys -p "synth_ecp5 -top top -json $@" $<

%_out.config: %.json
	nextpnr-ecp5 $(DEVICE) $(PACKAGE) --json $< --lpf $(LPF) --textcfg $@

%.bit: %_out.config
	ecppack --svf-rowsize 100000 --svf $(PROJ).svf $< $@

prog: $(PROJ).svf
	openocd -f /usr/share/trellis/misc/openocd/ecp5-versa5g.cfg \
	  -c "transport select jtag; init; svf $<; exit"

clean:
	rm -f *.json *_out.config *.bit *.svf
\end{lstlisting}

\subsection{CVE2 SoC Project}
Recommended structure (already provided under \texttt{/home/kaushal/cve2\_fpga\_project}):
\begin{lstlisting}[style=bashstyle]
cve2_fpga_project/
  rtl/           # SoC wrapper, stubs
  build/         # Generated outputs
  scripts/       # sv2v, yosys scripts
  firmware/      # RISC-V firmware (optional)
  versa.lpf      # Constraints
  Makefile       # Full flow: sv2v -> yosys -> nextpnr -> ecppack -> openocd
\end{lstlisting}

\section{Example 1: LED Blinker (Sanity Test)}
\subsection{Verilog Top}
\begin{lstlisting}[style=verilogstyle]
module top(
  input  wire       clk,
  output wire [7:0] led,
  output wire [13:0] disp
);
  // Slow counter to make LEDs visible
  reg [27:0] cnt = 28'd0;
  always @(posedge clk) cnt <= cnt + 1'b1;

  // Active-LOW outputs on Versa 5G
  assign led  = ~cnt[27:20];
  assign disp = ~14'h3FFF; // all OFF
endmodule
\end{lstlisting}

\subsection{Constraints (LPF)}
Use the official Versa 5G pins (as tested):
\begin{lstlisting}[style=bashstyle]
LOCATE COMP "clk" SITE "P3"; IOBUF PORT "clk" IO_TYPE=LVDS;
# LEDs
LOCATE COMP "led[0]" SITE "E16"; IOBUF PORT "led[0]" IO_TYPE=LVCMOS25;
LOCATE COMP "led[1]" SITE "D17"; IOBUF PORT "led[1]" IO_TYPE=LVCMOS25;
LOCATE COMP "led[2]" SITE "D18"; IOBUF PORT "led[2]" IO_TYPE=LVCMOS25;
LOCATE COMP "led[3]" SITE "E18"; IOBUF PORT "led[3]" IO_TYPE=LVCMOS25;
LOCATE COMP "led[4]" SITE "F17"; IOBUF PORT "led[4]" IO_TYPE=LVCMOS25;
LOCATE COMP "led[5]" SITE "F18"; IOBUF PORT "led[5]" IO_TYPE=LVCMOS25;
LOCATE COMP "led[6]" SITE "E17"; IOBUF PORT "led[6]" IO_TYPE=LVCMOS25;
LOCATE COMP "led[7]" SITE "F16"; IOBUF PORT "led[7]" IO_TYPE=LVCMOS25;
# 14-seg
LOCATE COMP "disp[0]" SITE "M20"; IOBUF PORT "disp[0]" IO_TYPE=LVCMOS25;
LOCATE COMP "disp[1]" SITE "L18"; IOBUF PORT "disp[1]" IO_TYPE=LVCMOS25;
LOCATE COMP "disp[2]" SITE "M19"; IOBUF PORT "disp[2]" IO_TYPE=LVCMOS25;
LOCATE COMP "disp[3]" SITE "L16"; IOBUF PORT "disp[3]" IO_TYPE=LVCMOS25;
LOCATE COMP "disp[4]" SITE "L17"; IOBUF PORT "disp[4]" IO_TYPE=LVCMOS25;
LOCATE COMP "disp[5]" SITE "M18"; IOBUF PORT "disp[5]" IO_TYPE=LVCMOS25;
LOCATE COMP "disp[6]" SITE "N16"; IOBUF PORT "disp[6]" IO_TYPE=LVCMOS25;
LOCATE COMP "disp[7]" SITE "M17"; IOBUF PORT "disp[7]" IO_TYPE=LVCMOS25;
LOCATE COMP "disp[8]" SITE "N18"; IOBUF PORT "disp[8]" IO_TYPE=LVCMOS25;
LOCATE COMP "disp[9]" SITE "P17"; IOBUF PORT "disp[9]" IO_TYPE=LVCMOS25;
LOCATE COMP "disp[10]" SITE "N17"; IOBUF PORT "disp[10]" IO_TYPE=LVCMOS25;
LOCATE COMP "disp[11]" SITE "P16"; IOBUF PORT "disp[11]" IO_TYPE=LVCMOS25;
LOCATE COMP "disp[12]" SITE "R16"; IOBUF PORT "disp[12]" IO_TYPE=LVCMOS25;
LOCATE COMP "disp[13]" SITE "R17"; IOBUF PORT "disp[13]" IO_TYPE=LVCMOS25;
\end{lstlisting}

\subsection{Build and Program}
\begin{lstlisting}[style=bashstyle]
yosys -p "synth_ecp5 -top top -json top.json" top.v
nextpnr-ecp5 --um5g-45k --package CABGA381 --json top.json --lpf versa.lpf --textcfg top_out.config
ecppack --svf-rowsize 100000 --svf top.svf top_out.config top.bit
openocd -f /usr/share/trellis/misc/openocd/ecp5-versa5g.cfg -c "transport select jtag; init; svf top.svf; exit"
\end{lstlisting}
If LEDs count and MSB toggles slowly, your board and toolchain are good.

\section{Example 2: CVE2 RISC-V SoC}
\subsection{Overview}
We use the OpenHW CVE2 core (RV32E), internal 8 KB memory, and a GPIO register mapped at \texttt{0x8000\_0000}. Firmware increments an 8-bit counter and writes to GPIO, driving the board LEDs.

\subsection{Flow Summary}
\begin{enumerate}
  \item Convert SystemVerilog to Verilog (sv2v)
  \item Synthesize with Yosys (\texttt{synth\_ecp5})
  \item Place and route with nextpnr-ecp5
  \item Pack bitstream with ecppack
  \item Program with OpenOCD
\end{enumerate}

\subsection{Run the Provided Project}
Already set up at \texttt{/home/kaushal/cve2\_fpga\_project}:
\begin{lstlisting}[style=bashstyle]
cd /home/kaushal/cve2_fpga_project
make -j1 all prog
\end{lstlisting}
This invokes sv2v/Yosys/nextpnr/ecppack/OpenOCD with the correct device, package, and constraints.

\subsection{Key Files (Concepts)}
\begin{itemize}
  \item SoC wrapper: connects CVE2 instruction/data busses to internal RAM and GPIO
  \item LPF constraints: use the same pin map as the blinker
  \item Makefile: automates the entire flow
\end{itemize}

\section{Troubleshooting (Most Common Issues)}
\subsection{OpenOCD cannot find the board}
\textbf{Symptom:} JTAG scan fails or no device found.\\
\textbf{Fix:}
\begin{itemize}
  \item Use the correct config: \texttt{/usr/share/trellis/misc/openocd/ecp5-versa5g.cfg}
  \item Try root: \texttt{sudo openocd ...} (as a quick test)
  \item Check cables and board power
\end{itemize}

\subsection{LEDs stuck ON (or OFF)}
\textbf{Symptom:} Outputs inverted.\\
\textbf{Fix:} LEDs are active-LOW on Versa 5G. Invert outputs in RTL: \texttt{assign led = ~value;}.

\subsection{Place-and-route fails timing}
\textbf{Symptom:} nextpnr timing errors.\\
\textbf{Fix:}
\begin{itemize}
  \item Reduce clock speed (use slower enables)
  \item Register long combinational paths (e.g., instruction/data reads)
  \item Ensure all clocks route through global nets
\end{itemize}

\subsection{Synthesis errors with SystemVerilog}
\textbf{Symptom:} Yosys rejects SV constructs.\\
\textbf{Fix:} Run sv2v to convert SV to Verilog first; simplify interfaces and packages.

\subsection{Nothing blinks}
\textbf{Checklist:}
\begin{itemize}
  \item Verify \texttt{versa.lpf} pins match the board
  \item Run the blinker first (sanity check)
  \item Confirm OpenOCD programming completes without errors
  \item Use a very slow counter to make activity visible
\end{itemize}

\section{Appendix A: Makefile (Blinker)}
\begin{lstlisting}[style=bashstyle]
PROJ = top
DEVICE = --um5g-45k
PACKAGE = --package CABGA381
LPF = versa.lpf

all: $(PROJ).bit

%.json: %.v
	yosys -p "synth_ecp5 -top top -json $@" $<

%_out.config: %.json
	nextpnr-ecp5 $(DEVICE) $(PACKAGE) --json $< --lpf $(LPF) --textcfg $@

%.bit: %_out.config
	ecppack --svf-rowsize 100000 --svf $(PROJ).svf $< $@

prog: $(PROJ).svf
	openocd -f /usr/share/trellis/misc/openocd/ecp5-versa5g.cfg \
	  -c "transport select jtag; init; svf $<; exit"

clean:
	rm -f *.json *_out.config *.bit *.svf
\end{lstlisting}

\section{Appendix B: LPF (Versa 5G)}
\begin{lstlisting}[style=bashstyle]
LOCATE COMP "clk" SITE "P3"; IOBUF PORT "clk" IO_TYPE=LVDS;
# LEDs
LOCATE COMP "led[0]" SITE "E16"; IOBUF PORT "led[0]" IO_TYPE=LVCMOS25;
LOCATE COMP "led[1]" SITE "D17"; IOBUF PORT "led[1]" IO_TYPE=LVCMOS25;
LOCATE COMP "led[2]" SITE "D18"; IOBUF PORT "led[2]" IO_TYPE=LVCMOS25;
LOCATE COMP "led[3]" SITE "E18"; IOBUF PORT "led[3]" IO_TYPE=LVCMOS25;
LOCATE COMP "led[4]" SITE "F17"; IOBUF PORT "led[4]" IO_TYPE=LVCMOS25;
LOCATE COMP "led[5]" SITE "F18"; IOBUF PORT "led[5]" IO_TYPE=LVCMOS25;
LOCATE COMP "led[6]" SITE "E17"; IOBUF PORT "led[6]" IO_TYPE=LVCMOS25;
LOCATE COMP "led[7]" SITE "F16"; IOBUF PORT "led[7]" IO_TYPE=LVCMOS25;
# 14-seg
LOCATE COMP "disp[0]" SITE "M20"; IOBUF PORT "disp[0]" IO_TYPE=LVCMOS25;
LOCATE COMP "disp[1]" SITE "L18"; IOBUF PORT "disp[1]" IO_TYPE=LVCMOS25;
LOCATE COMP "disp[2]" SITE "M19"; IOBUF PORT "disp[2]" IO_TYPE=LVCMOS25;
LOCATE COMP "disp[3]" SITE "L16"; IOBUF PORT "disp[3]" IO_TYPE=LVCMOS25;
LOCATE COMP "disp[4]" SITE "L17"; IOBUF PORT "disp[4]" IO_TYPE=LVCMOS25;
LOCATE COMP "disp[5]" SITE "M18"; IOBUF PORT "disp[5]" IO_TYPE=LVCMOS25;
LOCATE COMP "disp[6]" SITE "N16"; IOBUF PORT "disp[6]" IO_TYPE=LVCMOS25;
LOCATE COMP "disp[7]" SITE "M17"; IOBUF PORT "disp[7]" IO_TYPE=LVCMOS25;
LOCATE COMP "disp[8]" SITE "N18"; IOBUF PORT "disp[8]" IO_TYPE=LVCMOS25;
LOCATE COMP "disp[9]" SITE "P17"; IOBUF PORT "disp[9]" IO_TYPE=LVCMOS25;
LOCATE COMP "disp[10]" SITE "N17"; IOBUF PORT "disp[10]" IO_TYPE=LVCMOS25;
LOCATE COMP "disp[11]" SITE "P16"; IOBUF PORT "disp[11]" IO_TYPE=LVCMOS25;
LOCATE COMP "disp[12]" SITE "R16"; IOBUF PORT "disp[12]" IO_TYPE=LVCMOS25;
LOCATE COMP "disp[13]" SITE "R17"; IOBUF PORT "disp[13]" IO_TYPE=LVCMOS25;
\end{lstlisting}

\section{Appendix C: OpenOCD Notes}
\begin{itemize}
  \item Standard config: \texttt{/usr/share/trellis/misc/openocd/ecp5-versa5g.cfg}
  \item Useful options: \texttt{-d} (debug), \texttt{init; svf file.svf; exit}
  \item For SPI flash programming: use \texttt{ecppack --flash ...} and appropriate OpenOCD scripts (beyond scope here)
\end{itemize}

\section{Appendix D: Reading Reports}
\begin{itemize}
  \item Yosys resource summary: shown after synth, or with \texttt{stat}
  \item nextpnr timing: look for Fmax and critical path in the log
  \item Resource utilization: LUT4, FF, BRAM, DSP counts in nextpnr summary
\end{itemize}

\section{You’ve Got This}
When in doubt, start from the blinker. Once it works, move to the SoC. Keep outputs visible (LEDs, 14-seg) and make small, reversible changes. This booklet is designed so you can rebuild the whole flow without any external help.

\end{document}
